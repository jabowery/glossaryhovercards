\documentclass[11pt]{article}
\usepackage{amsmath,amssymb}
\usepackage[margin=1in]{geometry}

\title{Minimal Core Glossary of Etter's Technical Vocabulary}
\author{by James Bowery relying on various LLMs to help "distill" it so don't blame Tom}
\date{}

\begin{document}
\maketitle

\section*{Conventions}

Each entry is defined, as far as possible, only in terms of other entries in this
\emph{minimal core} plus ordinary logical vocabulary. GA / proof-assistant
readings are indicated briefly at the end of each item.

\bigskip

\begin{description}

% --- IDENTITY, DISCRIMINATORS, WORLD ---

\item[\textbf{Discriminator}]
A \emph{discriminator} $x$ is a structural device that fixes a notion of
\textbf{Relative Identity} and thereby a \textbf{World}$(x)$: for any $y,z$,
the statement $x(y=z)$ is decided solely by how $y,z$ appear in the
\textbf{Structure Tables} and \textbf{Count Tables} that $x$ uses. In GA, a
discriminator is implemented as a chosen family of \textbf{Selections} and
\textbf{Linkings} (projection operators) acting on multivectors.

\item[\textbf{World of a Discriminator} $\mathrm{World}(x)$]
The \emph{world} of a \textbf{Discriminator} $x$ is the class of \textbf{Things}
that $x$ can distinguish from itself by some pattern in its \textbf{Structure
Tables}: 
\[
\mathrm{World}(x) = \{ t : \neg x(t=x)\},
\]
where $x(t=x)$ is \textbf{Relative Identity}. In GA, $\mathrm{World}(x)$ is the
set of multivectors not identified with the ``state'' of $x$ under the
projections determined by $x$.

\item[\textbf{Thing}]
A \emph{thing} $t$ is any entity that lies in the \textbf{World} of at least one
\textbf{Discriminator}: there exist $x,y$ with $\neg x(t=y)$ under
\textbf{Relative Identity}. Thus, ``thinghood'' is defined by participation in
some pattern of discrimination. In GA, a thing is any multivector that survives
non-trivial \textbf{Selection} for some discriminator.

\item[\textbf{Relative Identity} $x(y=z)$]
\emph{Relative identity} is the identity statement read ``$y$ and $z$ are the
same \emph{for} discriminator $x$''. Formally, $x(y=z)$ holds iff, in all
\textbf{Structure Tables} and \textbf{Count Tables} used by $x$, the rows and
columns containing $y$ and $z$ are interchangeable under the same
\textbf{Linkings} and \textbf{Selections}. In GA, $x(y=z)$ means that the
projection operators implementing $x$ send $y$ and $z$ to the same multivector.

\item[\textbf{Quine Identity $Q_R$}]
Given a single binary relation $R$ whose extension is encoded in a
\textbf{Structure Table}, the \emph{Quine identity} $Q_R$ is the \textbf{Relative
Identity} induced solely by the way elements appear in the rows and columns of
that table. Concretely, $Q_R(x=y)$ holds iff $x$ and $y$ occur in exactly the
same positions within the \textbf{Structure Table} (same incident rows and
columns) and remain interchangeable under all \textbf{Selections} and
\textbf{Linkings} generated from $R$. In GA, $Q_R$ identifies terms whose
associated $R$-operators act identically on all multivectors.

\item[\textbf{Intrinsic Identity (of a predicate / system)}]
The \emph{intrinsic identity} of a primitive predicate (or of an axiom system)
is the finest \textbf{Relative Identity} obtained by closing the relevant Quine
identities $Q_R$ under all \textbf{Linkings}, \textbf{Selections} and
\textbf{Structural Evolution Operators} that are allowed by the system. Two
\textbf{Things} are intrinsically identical if every such structurally permitted
transformation preserves $Q_R(x=y)$ for all predicates $R$ in the language. In
GA, intrinsic identity is the equivalence of terms generated by all definable
operators in the chosen model.

% --- TABLES, COUNTS, DENSITIES ---

\item[\textbf{Structure Table}]
A \emph{structure table} is a \textbf{Count Table} whose range has been
canonically indexed so that only structural relations between rows and columns
matter. It is the basic carrier of structure on which \textbf{Selections},
\textbf{Linkings}, \textbf{Hidings}, and \textbf{Structural Evolution Operators}
act. In GA, a structure table corresponds to a fixed choice of basis for
multivectors representing the same structural pattern.

\item[\textbf{Count Table}]
A \emph{count table} is an extension represented as rows, columns, and integer
counts (possibly \textbf{Negative Counts}), so that multiple identical
relationships are combined into a single row with a total count. Count tables
support \textbf{Empty Pairs}, \textbf{Interference}, and the normalization to
\textbf{Density Count Matrices}. In GA, a count table is a multivector whose
coefficients are these integer counts.

\item[\textbf{Density Count Matrix}]
A \emph{density count matrix} is a count matrix derived from a \textbf{Count
Table} by appropriate normalization (typically by total non-empty count), so
that it can be used to assign ``structural probabilities'' to link values. It is
the object that evolves under \textbf{Structural Evolution Operators} and is
constrained by \textbf{Structural Unitarity} and \textbf{Past--Future
Symmetry}. In GA, it is the density operator built from multivectors whose
coefficients come from count tables.

% --- NEGATIVITY, EMPTY PAIRS, INTERFERENCE ---

\item[\textbf{Negative Count}]
A \emph{negative count} is an entry $-n$ in a \textbf{Count Table}, understood
only as part of a pattern of \textbf{Empty Pairs} and \textbf{Interference}.
Negative counts cannot appear in isolation in a \textbf{Proper} structure, but
only in combinations that can be cancelled or transformed by
\textbf{Structural Evolution Operators}. In GA, a negative count is a negative
coefficient on a blade in the multivector.

\item[\textbf{Empty Pair}]
An \emph{empty pair} is a pair of rows in a \textbf{Count Table} with counts
$n$ and $-n$ that together form a net-zero contribution to all
\textbf{Selections}, \textbf{Linkings}, and \textbf{Shape}. A finite union of
empty pairs that yields net zero counts on all rows is an \emph{empty part}.
Adding an empty part to any structure does not change its \textbf{Relative
Identities}, \textbf{Worlds}, or \textbf{Shape}. In GA, an empty part is a
multivector that is identically zero, built from terms $B + (-B)$.

\item[\textbf{Interference}]
\emph{Interference} is the phenomenon that occurs when \textbf{Negative Counts}
and positive counts in a \textbf{Count Table} combine into \textbf{Empty
Pairs}, thereby changing the pattern of non-empty rows under
\textbf{Selections} and \textbf{Linkings} without altering \textbf{Shape}. In
GA, interference is just addition of multivectors in which some blades cancel.

% --- SHAPE, CONGRUENCE-NUMBER ---

\item[\textbf{Shape}]
The \emph{shape} of a \textbf{Structure Table} is the class of all tables that
can be obtained from it by permutations of rows, columns, and value indices
that preserve the pattern of \textbf{Count Tables}, \textbf{Empty Parts},
\textbf{Relative Identities}, and \textbf{Relation-numbers}. Thus, shape is
the global structural invariant of a table under all allowed relabelings. In
GA, shape is the orbit of a multivector under the automorphism group that
preserves all structural operations. (In Principia Mathematica this was \emph{incorrectly} called \textbf{Relation Number} which is, in geometric analogy, congruence rather than mere similarity of shapes.).

\item[\textbf{Relation-number}]
The \emph{relation-number} of a part of a \textbf{Structure Table} is the
number of distinct occurrences of that part in the same \textbf{Shape} when we
apply all permutations that preserve \textbf{Relative Identity} and the
\textbf{Count Table} structure. It measures structured recurrence of parts
within a shape. In GA, a relation-number is the size of the orbit of a
sub-multivector under the automorphisms that fix the ambient shape.

% --- SELECTION, LINKING, HIDING, TIME ---

\item[\textbf{Selection}]
A \emph{selection} is an operation that takes a \textbf{Structure Table} and
designates some of its rows and columns as \emph{foreground}, leaving the rest
as \emph{background}. Selections generate the frames of a \textbf{Structural
Time} sequence and determine which parts of a structure a \textbf{Discriminator}
actually uses to fix \textbf{Relative Identity}. In GA, selections are
projectors onto subspaces spanned by chosen blades.

\item[\textbf{Linking}]
\emph{Linking} is the operation that imposes equality conditions between
columns (or rows) of one or more \textbf{Structure Tables}, by deleting all
rows that violate these equalities and then re-indexing the resulting
\textbf{Count Table}. Linking thus creates joint structures and defines the
``present'' interface between past and future in \textbf{Structural Time}. In
GA, linking is projection onto the subspace where linked indices coincide.

\item[\textbf{Hiding}]
\emph{Hiding} is the operation that removes columns or rows that have been used
in \textbf{Linking} but are no longer of interest to a given
\textbf{Discriminator}. Hiding preserves \textbf{Relative Identity} on the
remaining variables while marginalizing out the hidden ones, within the same
\textbf{Shape}. In GA, hiding corresponds to partial trace or projection that
forgets certain indices.

\item[\textbf{Structural Time}]
\emph{Structural time} is the ordering of \textbf{Selections}, \textbf{Linkings}
and \textbf{Hidings} applied to \textbf{Structure Tables} and \textbf{Density
Count Matrices}, rather than an external parameter. A \emph{dynamic sequence}
is a finite or infinite chain of such operations, possibly governed by a
\textbf{Structural Evolution Operator}. In GA, structural time is encoded by
the composition order of projectors and linear operators on multivectors.

% --- EVOLUTION, UNITARITY, PAST-FUTURE SYMMETRY ---

\item[\textbf{Structural Evolution Operator $T$}]
A \emph{structural evolution operator} $T$ is a map that sends \textbf{Structure
Tables} and their corresponding \textbf{Density Count Matrices} to new ones in a
way that respects \textbf{Shape}, \textbf{Relative Identity}, and the
\textbf{Empty Part} bookkeeping. When $T$ is invertible and satisfies
\textbf{Structural Unitarity}, it generates reversible \textbf{Structural Time}.
In GA, $T$ is a linear operator (often unitary) acting on the space of
multivectors and their density operators.

\item[\textbf{Structural Unitarity}]
A \emph{structurally unitary} evolution is one in which a \textbf{Structural
Evolution Operator} $T$ preserves the total ``count-mass'' and the normalization
of \textbf{Density Count Matrices}, i.e.\ it preserves all structural
probabilities. Structural unitarity thus ties the reversible part of
\textbf{Structural Time} to invariants of \textbf{Shape} and \textbf{Relative
Identity}. In GA, structural unitarity is realized by norm-preserving linear
operators (unitary or orthogonal) on the state space.

\item[\textbf{Past--Future Symmetry}]
A system has \emph{past--future symmetry} when its \textbf{Structural Time}
evolution admits a \textbf{Structural Evolution Operator} $T$ and an inverse
$T^{-1}$ such that the corresponding \textbf{Density Count Matrices} and
\textbf{Relative Identities} are invariant under reversing the order of
applications of $T$ (modulo \textbf{Empty Parts}). In this domain, quantum-like
\textbf{Interference} and \textbf{Structural Unitarity} dominate. In GA, this
is realized when the dynamics is given by a unitary family $T(t)$ and its
adjoint $T(t)^{-1}$, with symmetry under time reversal.

\end{description}

\end{document}
